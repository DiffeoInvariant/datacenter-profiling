
This project provides a low-overhead\footnote{Even if the data collecting programs have generated over 20,000 total entries, the program uses less than 500 kB of network data (i.e. transmits < 500 kB over the network).} option to profile the network usage of an MPI program and transmit the results to a remote user via a REST API (including browser-friendly HTML). It does not require the user to recompile their code, only to run the data collection programs and then the server before running their program. This could be done explicitly on the command-line or in a batch script, running the data collection programs and server in the background. Furthermore, even an inexperienced user should be able to extend this code to collect more detailed data if so desired. However, because eBPF programs can currently only be run as root, the user needs root access to use this software in its current form, which renders it unusable on many HPC clusters. If one day eBPF programs can be run as a non-root user, this software will be able to be run on essentially arbitrary clusters via the provided Singularity container. Currently, the provided container works in case the user doesn't want to install the relevant dependencies on their machine (which can be done automatically on debian-based systems with a script we wrote, \lstinline{ubuntu-install.sh}). 
